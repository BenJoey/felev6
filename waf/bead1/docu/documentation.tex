\documentclass[11pt,a4paper]{article}
\usepackage[utf8]{inputenc}
\usepackage[T1]{fontenc}
\usepackage{amsmath}
\usepackage[usenames,dvipsnames,svgnames,table]{xcolor}
\usepackage[normalem]{ulem}
\usepackage[left=1.00cm, right=1.00cm, top=0.40cm, bottom=0.2cm]{geometry}
\PassOptionsToPackage{defaults=hu-min}{magyar.ldf}
\usepackage[magyar]{babel}
\usepackage{framed, fancyhdr, wasysym, graphicx, multirow}

\begin{document}
\renewcommand{\labelitemi}{\textbullet}
\def\br{\\[0.1cm]}
\thispagestyle{empty}
\begin{center}
	\colorbox{lightgray}{{\large JPSMA3} \hspace{3cm} {\large Webes alkalmazások fejlesztése 1. beadandó} \hspace{5cm} \thepage}
\end{center}
\begin{framed}
	\begin{flushleft}
		{\large \textbf{Bauer Bence}}
		\hspace{3cm}{\large \textbf{1.Beadandó/9.Feladat}}
		\hspace{5.4cm}{\large 2019.03.13.}\br
		{\large JPSMA3}\br
		{\large bauerbence5@gmail.com}
	\end{flushleft}
\end{framed}
\section{Feladat}
Készítsünk egy mozi üzemeltető rendszert, amely alkalmas az előadások, illetve
jegyvásárlások kezelésére.
1. részfeladat: a webes felületen keresztül a nézők tekinthetik meg a moziműsort,
valamint rendelhetnek jegyeket.
A főoldalon megjelenik a napi program, azaz mely filmeket mikor vetítik a
moziban, valamint kiemelve az öt legfrissebb (legutoljára felvitt) film plakátja.
A filmet kiválasztva megjelenik annak részletes leírása (rendező, főszereplők,
hossz, szinopszis), plakátja, továbbá az összes előadás időpontja.
Az időpontot kiválasztva lehetőség nyílik helyfoglalásra az adott előadásra.
Ekkor a felhasználónak meg kell adnia a lefoglalandó ülések helyzetét (sor,
illetve oszlop) egy, a mozitermet sematikusan ábrázoló grafikus felületen.
Egyszerre legfeljebb 6 jegy foglalható, és természetesen csak a szabad helyek
foglalhatóak (amelyek nem foglaltak, vagy eladottak). A felhasználónak ezen
felül meg kell adnia teljes nevét, valamint telefonszámát, ezzel véglegesíti a
foglalást.
\end{document}